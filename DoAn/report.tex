\documentclass{article}

\usepackage[a4paper, total={6in, 10in}]{geometry}
\usepackage[utf8]{vietnam}
\usepackage{cite}

\begin{document}
\title{
      \LARGE{
            \textbf{
                  Xây dựng mô hình Machine Learning \\ dự báo cháy rừng ở các tỉnh Tây Nguyên \\ dựa vào dữ liệu lịch sử thời tiết.
            }
      }
}

\author{
      Nguyễn Đại Kỳ\\
      19521731\\
      \and
      Văn Viết Hiếu Anh\\
      19521225
      \and
      Lê Văn Phước\\
      19522054
}

\date{\today} % Date for the report
\maketitle % Insert the title, author and date
\begin{center}
      \begin{tabular}{l l}
            Môn học:              & CS114 - Máy học       \\
            \\
            Giảng viên hướng dẫn: & Lê Đình Duy           \\
                                  & Phạm Nguyễn Trường An \\
      \end{tabular}
\end{center}

\tableofcontents

\pagebreak

%----------------------------------------------------------------------------------------
%	SECTION 1
%----------------------------------------------------------------------------------------

\section{Abstract}

To build a model that can forecasting the warning level of wildfire for each invidual place in Western Highlands, Vietnam.


\begin{description}
      \item[Problem]
            The relationship between the relative quantities of substances taking part in a reaction or forming a compound, typically a ratio of whole integers.
      \item[Purpose]
            The mass of an atom of a chemical element expressed in atomic mass units. It is approximately equivalent to the number of protons and neutrons in the atom (the mass number) or to the average number allowing for the relative abundances of different isotopes.
      \item[Methods]
            The mass of an atom of a chemical element expressed in atomic mass units. It is approximately equivalent to the number of protons and neutrons in the atom (the mass number) or to the average number allowing for the relative abundances of different isotopes.
      \item[Result]
            The mass of an atom of a chemical element expressed in atomic mass units. It is approximately equivalent to the number of protons and neutrons in the atom (the mass number) or to the average number allowing for the relative abundances of different isotopes.
\end{description}

%----------------------------------------------------------------------------------------
%	SECTION 2
%----------------------------------------------------------------------------------------

\section{Introduction}
To build a model that can forecasting the \cite{website:fermentas-lambda} warning level
of wildfire for each invidual place in Western Highlands, Vietnam

%----------------------------------------------------------------------------------------
%	SECTION 3
%----------------------------------------------------------------------------------------

\section{Dataset}

\subsection{Source of Data}
\subsubsection{Weather Data}
\paragraph{weather.com}

\paragraph{worldweatheronline.com}

\subsubsection{Fire Data}
firewatchvn.kiemlam.org.vn

\subsection{Imputation of Data}
\subsection{Creation of Dataset}




%----------------------------------------------------------------------------------------
%	SECTION 4
%----------------------------------------------------------------------------------------

\section{Methods}

\subsection{Convolutional neural network}
\subsection{Fully-connected neural network}


%----------------------------------------------------------------------------------------
%	SECTION 5
%----------------------------------------------------------------------------------------

\section{Conclusion}


%----------------------------------------------------------------------------------------
%	BIBLIOGRAPHY
%----------------------------------------------------------------------------------------

\bibliographystyle{plain}
\bibliography{source}

%----------------------------------------------------------------------------------------

\end{document}